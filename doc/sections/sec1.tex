\section{Implementing Maze Solver in Lisp}
For the implementation, we are using 
\href{https://en.wikipedia.org/wiki/Common_Lisp}
{\textbf{Common Lisp}}. Common Lisp (CL) is a dialect of the 
Lisp programming language, published in ANSI standard.
\subsection{Map representation}
For map representation, a simple binary matrix is 
enough. The 0s are points that agents can take as 
a path, and 1s are obstacles. For convenience, the 
program in Listing \ref{lis:get-file} is written to get the map from a file 
and change it to a 2D-array. Take note, the global variables start and end
represent our starting and end points.
\begin{lstlisting}[language=Lisp, style=mystyle,
                 caption=Getting map from a file and represent it as 2D-array,
                 label=lis:get-file]
(defun delimiterp (c) (or (char= c #\Space) (char= c #\,)))
(defun my-split (string &key (delimiterp #'delimiterp))
  (loop :for beg = (position-if-not delimiterp string)
    :then (position-if-not delimiterp string :start (1+ end))
    :for end = (and beg (position-if delimiterp string :start beg))
    :when beg :collect (subseq string beg end)
    :while end))
;; *maze* variable for storing our map az matrix
(setq *maze* nil)
;;define start and end points here
(setq *start* '(0 0))
(setq *end* '(3 0))
;; reading from the fiel maze.txt
(let ((in (open "maze.txt" :if-does-not-exist nil)))
   (when in
      (loop for line = (read-line in nil)
      
      while line do (push  (mapcar #'parse-integer (my-split line)) *maze*))
      (close in)
   )
)
;; function for converting 2d list to 2d-array
(defun list-to-2d-array (list)
  (make-array (list (length list)
                    (length (first list)))
              :initial-contents list))
;; shape size of the maze
(defconstant N (length *maze*))
(defconstant M (length (first *maze*)))
;; reverse tge *maze* and convert to the array
(setf *maze* (list-to-2d-array (reverse *maze*)))
\end{lstlisting}
Figure \ref{fig:file-content} shows an example, how we can
define our file content.
\begin{figure}[H]
\centering
\begin{tabular}{c}
    
\begin{lstlisting}[ ]
0 0 0 0 0 
1 1 0 1 0 
1 0 0 1 0
0 0 1 1 0
\end{lstlisting}
\end{tabular}    
\caption{Example of file content}
\label{fig:file-content}
\end{figure}

\subsection{Getting accessible neighbours}
Listing \ref{lis:get-neighbor} shows a implementation of
a helper function to get the accessible neighbours from a
point that is represents by a list (x y). The function also takes
a hash-table path to check if the neighbour is 
visited before or not. If the neighbour grid is in the boundry
of map , is not a obstacle and is not visted before, it will be
added to the list of the accessible neighbours. And it will be 
returned by the function.


\begin{lstlisting}[language=Lisp, style=mystyle,
                 caption=Finding accessible neighbours from a specific point,
                 label=lis:get-neighbor]
; helper function for finding adjacency grid that we can go from index
;; with respect to the path that we've visted
(defun get-neighbors(index path)
    (let ((neighbors nil))
    ;; check up
    (let (( up (list (- (nth 0 index) 1) (nth 1 index)) )) 
        (if (and (>= (nth 0 up) 0)                       ; if the index is in map
            (equal (aref *maze* (nth 0 up) (nth 1 up)) 0); there is no obstacle
            (null (gethash up path))                     ; it's not visited before
        )
          (push (list up 'U) neighbors)
        )
    )
    ;; check right
    (let (( right  (list (nth 0 index) (+ (nth 1 index) 1)) ))
        (if (and (< (nth 1 right) M)
            (equal (aref *maze* (nth 0 right) (nth 1 right)) 0)
            (null (gethash right path))
        )
        (push (list right 'R) neighbors)
        )
    )
    ;; check below
    (let (( down  (list (+ (nth 0 index) 1) (nth 1 index)) ))
        (if (and (<  (nth 0 down)  N)
            (equal (aref *maze* (nth 0 down) (nth 1 down)) 0)
            (null (gethash down path))
        )
        (push (list down 'D) neighbors)
        )
    )
    ;; check left
    (let (( left (list (nth 0 index) (- (nth 1 index) 1)))) 
        (if (and (>= (nth 1 left) 0) 
            (equal (aref *maze* (nth 0 left) (nth 1 left)) 0)
            (null (gethash left path))
        )
          (push (list left 'L) neighbors)
        )
    )
    (return-from get-neighbors neighbors);return all possible neighbor accessible from index
    )
)
\end{lstlisting}

\subsection{Solving the Maze}
Now, we reach our main function for solving 
the maze. This function takes the map, its 
starting-point, and end-point. With the 
help of the \textbf{get-neighbors} function, 
it will solve the maze by returning the plan. 
The plan consists of the direction that the 
agent needs to take at each point in the path 
to the goal.

\begin{lstlisting}[language=Lisp, style=mystyle,
                 caption=main function for solving the maze,
                 label=lis:solve-maze]
;; solve the maze by getting 2d-array as map
;; and start as (x y) that specifies the staring point
;; and end for end point
(defun solve-maze(maze start end)
    ;; path hash-table for stroing the path we take to reach to each grid
    ;; path[p1] = (p0 L) says that we reach to p1 from p0 by going to the left
    (setq path (make-hash-table :test 'equal) )
    ;; stack for visiting grid recursively
    (setq stack nil) 
    (push start stack)
    ;; start 
    (setf (gethash start path) (list -1 -1))
    (let ((p
    (loop
        ;; taking the first element to expand its accessible neighbours
        (setq el  (car stack))
        (setf stack (cdr stack))
        ;; the loop finish when the el is the end point
        ;; or the stack is empty and that means there is no path to goal
        (when (or (null el) (equal el end)) (return path))
        ;; geting el's neighbors
        (let ((ns (get-neighbors el path) ))
            (if (not (equal ns nil))
                (dolist (neighbor ns)
                (progn
                    ;; ading the neighbor to the stack
                    ;; and to the path
                    (push (car neighbor) stack)
                    (setf (gethash (car neighbor) path) (list el (cadr neighbor)))
                )
                )
                
            )
        
        )
    ) ))
    ;; if there is a path to end point
    ;; it we'll print the plan
    (if (gethash end path)
        (progn
            (setq  temp end)
            (setq plan nil)
            (loop (when (equal temp start) (return  plan))
                (push (cadr (gethash temp path)) plan)
                
                (setf temp (car (gethash temp path)))
            )
        )
        (format t "no path is found!~%")
    )
    )
) 
\end{lstlisting}   

\subsection{Running the program}
Now that the main function is completed we can run 
our program on some inputs. The code in Listing \ref{lis:final}
shows how with the help of the \textbf{solve-maze} and
map, starting-point and end-point we can write our final
part of our program.

\begin{lstlisting}[language=Lisp, style=mystyle,
                 caption=final part of program,
                 label=lis:final]
       ;; printing the map
(format t "Map:~%")        
(dotimes (x 4)
    (dotimes(y 5)
        (format t "~a " (aref *maze* x y))
    )
    (format t "~%")
)
(terpri)
(format t "start:~a ~%" *start*) 
(format t "Goal:~a ~%" *end*) 
(terpri)
;; solve the with starting point (0 0) and end point of (3 0)
(format t "Plan to find the goal:~% ~a" (solve-maze *maze* *start* *end*))          
\end{lstlisting}
Finally, you can see an example of the exection of the program in the
Listing \ref{lis:execute}. 
\begin{lstlisting}[language=Bash,
caption= Exectuion of the program,
label=lis:execute,
backgroundcolor=\color{backcolour},
keywordstyle=\color{magenta},
otherkeywords={clisp},
emph={$},
emphstyle={\color{deepblue}\ttfamily},
]
$ cat maze.txt
0 0 0 0 0 
1 1 0 1 0 
1 0 0 1 0
0 0 1 1 0
$ clisp maze_solver.lisp
Map:
0 0 0 0 0 
1 1 0 1 0 
1 0 0 1 0 
0 0 1 1 0 

start:(0 0) 
Goal:(3 0) 

Plan to find the goal:
 (R R D D L D L)
\end{lstlisting}